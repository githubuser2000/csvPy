\documentclass[12pt]{article}
\usepackage[utf8]{inputenc}
\usepackage{amsmath, amssymb}

\title{Ordinalzahlen, Mächtigkeit und Folgen}
\author{Alexander Kern}
\date{}

\begin{document}

\maketitle

\section{Sind gewichtete p-adische Zahlenteile mit Gewichten 0--100\% wie Ordinalzahlen?}

Gewichtete p-adische Zahlenteile mit Gewichten in $[0,1]$ (oder 0--100\%) beschreiben die \emph{relative Relevanz oder Tiefe} eines Teils in einer p-adischen Struktur. 

\begin{itemize}
    \item Ordinalzahlen messen \emph{Positionen in wohlgeordneten Mengen}, nicht Maß oder Anteil.
    \item Prozentwerte beschreiben \emph{stetige Gewichtung}, Ordinale sind \emph{diskret und wohlgeordnet}.
    \item Daher können gewichtete p-adische Teile nicht direkt als Ordinalzahlen interpretiert werden.
\end{itemize}

Man kann sie formal als \emph{maßtheoretische oder informationsbasierte Bewertung} eines Elements in einem ultrametrischen Raum verstehen.

\[
\text{Gewichtetes Element } x_i \in \mathbb{Z}_p \text{ mit Gewicht } w_i \in [0,1]
\]

Diese Gewichte sagen \emph{nicht} aus, wie mächtig die Menge ist, sondern nur, wie relevant oder dominant $x_i$ innerhalb der Struktur ist.

\bigskip

\section{Ist die Mächtigkeit einer Menge eine Menge oder eine Klasse?}

Die Antwort hängt von der Größe der Menge ab:

\begin{itemize}
    \item \textbf{Kleine Mengen:} Endliche Mengen oder Mengen wie $\mathbb{N}, \mathbb{R}$  
    → Ihre Mächtigkeit (Kardinalzahl) ist \emph{eine Menge}, z.B. $|\mathbb{N}| = \aleph_0$.
    \item \textbf{Große Mengen / Klassen:} Die Menge aller Mengen $V$  
    → Ihre „Mächtigkeit“ existiert nicht als Menge, sondern nur als \emph{echte Klasse}.
\end{itemize}

\[
\begin{array}{c|c|c}
\text{Objekt} & \text{Kardinalität} & \text{Bemerkung} \\
\hline
\text{Endliche Menge} & n \in \mathbb{N} & \text{Menge} \\
\text{Abzählbare Menge } \mathbb{N} & \aleph_0 & \text{Menge} \\
\text{Überabzählbare Menge } \mathbb{R} & \mathfrak{c} & \text{Menge} \\
\text{Klasse aller Mengen } V & - & \text{echte Klasse} \\
\end{array}
\]

\bigskip

\section{Begrenzen Ordinalzahlen divergente oder konvergente Zahlen?}

\subsection{Grundprinzip}
\begin{itemize}
    \item Ordinalzahlen messen \emph{Positionen in wohlgeordneten Mengen}, nicht den Wert einer Zahl.
    \item Konvergenz oder Divergenz ist ein \emph{metrisches Konzept} in $\mathbb{R}$ oder $\mathbb{C}$.
    \item Ordinale können \emph{indirekt} eine Folge begrenzen, wenn sie als Indexmenge dienen, z.B.:
\end{itemize}

\subsection{Beispiele}

\textbf{Klassische Folge:}
\[
x_n = n \quad \text{divergiert in } \mathbb{R}
\]
\[
x_n = \frac{1}{n} \quad \text{konvergiert in } \mathbb{R}
\]

\textbf{Transfinite Folge:}
\[
x_\alpha = \frac{1}{\alpha+1}, \quad \alpha < \omega_1
\]
Hier indiziert $\alpha$ Ordinalzahlen bis $\omega_1$; das Limit ist
\[
\lim_{\alpha \to \omega_1} x_\alpha = 0
\]

\subsection{Fazit}

\begin{itemize}
    \item Ordinale \emph{begrenzen nur die Indexmenge} der Folge.
    \item Sie erzwingen \emph{nicht die Konvergenz} der Zahlenwerte.
    \item Merksatz:
    \[
    \text{Ordinal} \neq \text{Konvergenzwert}, \quad
    \text{Konvergenz} \text{ wird durch Metrik bestimmt, nicht durch Ordnung.}
    \]
\end{itemize}

\end{document}

