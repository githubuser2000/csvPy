\documentclass[11pt]{article}

\usepackage[T1]{fontenc}
\usepackage[utf8]{inputenc}
\usepackage{lmodern}
\usepackage{amsmath, amssymb, amsthm}
\usepackage{geometry}
\geometry{margin=2.5cm}

\title{Darstellungsformen eines geordneten Körpers}
\author{}
\date{}

\begin{document}
\maketitle

\section*{0. Ausgangspunkt: Was ist ein geordneter Körper?}

Ein \emph{geordneter Körper} ist eine mathematische Struktur
\[
(K,\; +,\; \cdot,\; 0,\; 1,\; <),
\]
wobei
\begin{itemize}
  \item $(K,+,\cdot,0,1)$ ein Körper ist,
  \item $<$ eine totale Ordnung auf $K$ ist,
  \item die Ordnung mit Addition und Multiplikation verträglich ist.
\end{itemize}

Wichtig ist: Ein geordneter Körper ist primär ein \emph{Strukturbegriff}, kein
Objekt einer bestimmten ontologischen Kategorie wie Zahl oder Menge.

\section*{1. Darstellung als Zahl (arithmetische Kodierung)}

In der Metamathematik kann ein geordneter Körper durch eine einzelne natürliche
Zahl kodiert werden, etwa mittels Gödelnummerierung.

\[
\ulcorner (K,+,\cdot,0,1,<) \urcorner \;\in\; \mathbb{N}.
\]

Dabei gilt:
\begin{itemize}
  \item Symbole, Formeln und Axiome werden numerisch kodiert,
  \item die gesamte Theorie des geordneten Körpers wird auf eine Zahl abgebildet.
\end{itemize}

Schematisch:
\[
\text{OrdFld} \;=\; 2^{n_1}\cdot 3^{n_2}\cdot 5^{n_3}\cdots
\]

\paragraph{Bedeutung.}
Diese Darstellung ist vollständig syntaktisch. Die algebraische und ordnungstheoretische
Struktur ist nicht direkt sichtbar, sondern nur implizit über die Kodierung gegeben.

\section*{2. Darstellung als Menge (ZFC-Standard)}

In der üblichen mengentheoretischen Grundlegung wird ein geordneter Körper als
Menge mit Zusatzstruktur realisiert.

\[
(K,+,\cdot,<)
\]

wobei
\begin{align*}
+ &\subseteq K \times K \times K,\\
\cdot &\subseteq K \times K \times K,\\
< &\subseteq K \times K.
\end{align*}

Eine mögliche formale Kodierung ist etwa:
\[
\bigl\{ \{0,K\},\{1,+\},\{2,\cdot\},\{3,<\} \bigr\}.
\]

\subsection*{Beispiel: Die rationalen Zahlen}

\[
\mathbb{Q}
=
\{(a,b)\in\mathbb{Z}\times\mathbb{Z}_{\neq 0}\}/\sim
\]

mit den üblichen Definitionen von $+$, $\cdot$ und $<$ als Mengen von Tupeln.

\paragraph{Bedeutung.}
Diese Darstellung ist mathematisch präzise und operativ, aber nicht kanonisch:
Isomorphe Körper sind im Allgemeinen nicht gleich als Mengen.

\section*{3. Darstellung als Klasse (strukturelle Sicht)}

\subsection*{3.1 Klasse aller geordneten Körper}

\[
\mathsf{OrdFld}
=
\{ (K,+,\cdot,<) \mid K \text{ ist ein geordneter Körper} \}.
\]

Dies ist im Allgemeinen eine echte Klasse, keine Menge.

\subsection*{3.2 Isomorphietyp eines Körpers}

Für einen konkreten Körper, etwa $\mathbb{R}$:
\[
[\mathbb{R}]
=
\{ K \mid K \cong \mathbb{R} \text{ als geordneter Körper} \}.
\]

\subsection*{3.3 Kategorientheoretische Sicht}

In der Kategorie der geordneten Körper
\[
\mathbf{OrdFld}
\]
ist ein Körper kein singuläres Objekt im ontologischen Sinn, sondern ein
Isomorphietyp innerhalb der Kategorie.

\paragraph{Bedeutung.}
Die Klassendarstellung abstrahiert vollständig von konkreten Implementierungen
und erfasst die Struktur „bis auf Isomorphie“.

\section*{4. Vergleich der drei Darstellungsformen}

\[
\begin{array}{l|l|l}
\text{Form} & \text{Schreibweise} & \text{Status} \\
\hline
\text{Zahl} & \ulcorner K \urcorner \in \mathbb{N} & \text{syntaktischer Code} \\
\text{Menge} & (K,+,\cdot,<) & \text{konkretes Objekt} \\
\text{Klasse} & \mathsf{OrdFld},\; [K] & \text{Strukturtyp}
\end{array}
\]

\section*{5. Zentrale Einsicht}

Der geordnete Körper selbst ist weder Zahl, noch Menge, noch Klasse.
Er ist ein \emph{Strukturbegriff}, der je nach metatheoretischem Rahmen
unterschiedlich realisiert wird:

\begin{itemize}
  \item als Zahl: syntaktische Repräsentation,
  \item als Menge: implementierte Struktur,
  \item als Klasse: semantischer Typ.
\end{itemize}

Dies ist direkt vergleichbar mit:
\begin{itemize}
  \item Programm als Bitstring,
  \item Programm als abstrakter Syntaxbaum,
  \item Programm als mathematische Funktion.
\end{itemize}

\section*{6. Kurzfassung}

Ein geordneter Körper kann als natürliche Zahl kodiert, als strukturierte Menge
implementiert oder als Klasse von isomorphen Modellen aufgefasst werden – keine
dieser Formen ist der geordnete Körper selbst, sondern jeweils eine Darstellung
auf einer anderen Abstraktionsebene.

\end{document}

