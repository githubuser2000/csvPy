\documentclass[12pt]{article}
\usepackage[utf8]{inputenc}
\usepackage{amsmath, amssymb}

\title{Geordnete Körper, CRC und RAID}
\author{Alexander Kern}
\date{}

\begin{document}

\maketitle

\section{Frage}
Eignet sich der geordnete Körper für Redundanz wie RAID und irgendwie für CRC?

\section{Antwort}

\subsection{1. Eigenschaften geordneter Körper}
Ein geordneter Körper \(K\) ist ein Körper mit einer totalen Ordnung \(<\), die mit der Addition und Multiplikation kompatibel ist:

\begin{itemize}
    \item Additivität: \(a,b \in P \Rightarrow a+b \in P\)
    \item Multiplikativität: \(a,b \in P \Rightarrow a\cdot b \in P\)
    \item Trichotomie: Für jedes \(a \in K\) gilt genau eine: \(a \in P\), \(-a \in P\), \(a = 0\)
\end{itemize}

Beispiele: \(\mathbb{Q}, \mathbb{R}\)  
Nicht-geordnet: \(\mathbb{C}\)

\subsection{2. Anforderungen von CRC und RAID}
\begin{itemize}
    \item \textbf{CRC}: arbeitet über endlichen Körpern, meist \(\mathbb{F}_2\) oder \(\mathbb{F}_{2^n}\), mit Polynomdivision modulo 2.
    \item \textbf{RAID}: z.B. RAID 5/6 nutzt XOR oder Reed-Solomon-Codes über \(\mathbb{F}_{2^8}\).
    \item Eigenschaften:
    \begin{itemize}
        \item Endliche Menge von Elementen
        \item Abgeschlossene Addition und Multiplikation
        \item Diskrete Struktur für deterministische Redundanz
    \end{itemize}
\end{itemize}

\subsection{3. Vergleich geordneter Körper vs. CRC/RAID}

\begin{tabular}{c|c|c}
\textbf{Eigenschaft} & \textbf{Geordneter Körper (\(\mathbb{R}\))} & \textbf{CRC/RAID} \\
\hline
Endlichkeit & ❌ unendlich & ✅ endlich \\
Addition & ✅ & ✅ \\
Multiplikation & ✅ & ✅ \\
Ordnung & ✅ relevant für Analysis & irrelevant \\
Diskrete Struktur & ❌ stetig & ✅ diskret, XOR-basiert \\
Modularität / Polynomrechnung & ❌ nicht geeignet & ✅ nötig \\
\end{tabular}

\subsection{4. Fazit}
\begin{itemize}
    \item Geordnete Körper sind geeignet für **Vergleich, Analysis und Optimierung**.
    \item CRC und RAID benötigen **endlich-algebraische Strukturen**, z.B. \(\mathbb{F}_2\) oder \(\mathbb{F}_{2^8}\).
    \item Geordnete Körper sind **nicht geeignet** für digitale Redundanz oder Fehlerkorrektur.
\end{itemize}

\end{document}

